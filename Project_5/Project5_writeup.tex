\documentclass{article}
\usepackage[margin=1in]{geometry}
\usepackage{graphicx}
\usepackage{framed}
\usepackage{listings}
\lstset{
    breaklines=true,
    breakatwhitespace=false}
\graphicspath{ {images/} }

\begin{document}
\title{ECE 4802 Project 5}
\author{Thomas Mackintosh}
\maketitle

\section{8.4}

In this exercise we want to identify primitive elements (generators) of a multiplicative group since they play a big role in the DHKE and and many other publickey schemes based on the DL problem. You are given a prime p = 4969 and the corresponding multiplicative group $\textbf{Z}_{4969}^*$.\\

1. Determine how many generators exist in $\textbf{Z}_{4969}^*$.
We can use the following code to determine the number of generators in $\textbf{Z}_{4969}^*$:
\begin{framed}

\begin{verbatim}
import math

gen_count = 0

for i in range(1,4969):
    temp = math.gcd(i,4968)
    if temp == 1:
        gen_count += 1

print(gen_count) #1584
\end{verbatim}

\end{framed}

This gives us a result of 1584 generators for $\textbf{Z}_{4969}^*$.\\

2. What is the probability of a randomly chosen element a $ \in \textbf{Z}_{4969}^*$ being a generator?\\
\begin{center}
$P(\textrm{a is a generator}) = 1584\div4969 = 0.3188$
\end{center}

3. Determine the smallest generator a $ \in \textbf{Z}_{4969}^*$ with $a > 1000$.\\

The following code finds the first generator a $ \in \textbf{Z}_{4969}^*$ where $a > 1000$. The code uses the squareMultiply function I create in a later section. pow() would also work in place of squareMultiply.

\begin{framed}
\begin{verbatim}
#generators.py
#An integer is a generator if integer ^ 4968 = 1 mod 4969

for a in range(1000,5000):
    temp = squareMultiply(a,4968,4968)
    if temp == 1:
        print(a)
        break
\end{verbatim}
\end{framed}

The code gives us a result of 1013 as the first generator $a > 1000$.\\

4. What measures can be taken in order to simplify the search for generators for
arbitrary groups $\textbf{Z}_{p}^*$?\\

There are $\phi(p-1)$ generators for $\textbf{Z}_{p}^*$. To make generators to be easy to find, we can choose a 'safe prime' $p=2q+1$ where q is prime since then $\phi(p-1)=\phi(2q)=q-1$. This would cause half the group to be generators compared to the current third.

\section{Square Multiply Algorithm} Implement the square and multiply algorithm using a computer language of your choice.\\

Below is my program which implements the square and multiply algorithm in Python:

\begin{framed}
\begin{verbatim}
#square_multiply.py
#Implements the square and multiply algorithm

def squareMultiply(b, e, m):
    r = 1
    while e > 0:

        if e % 2 == 1:
            r = (r * b) % m
        b = (b * b) % m
        e >>= 1
    return r

res = squareMultiply(275973,456789,583903)
print("For a = 235973; e = 456789; p = 583903, we get: ", res)

res = squareMultiply(984327457683,2153489582,994348472629)
print("For a = 984327457683; e = 2153489582; p = 994348472629, we get: ", res)

\end{verbatim}
\end{framed}

From this code we get the following output:

\begin{framed}
\begin{verbatim}
For a = 235973; e = 456789; p = 583903, we get:  277334
For a = 984327457683; e = 2153489582; p = 994348472629, we get:  331688688384
\end{verbatim}
\end{framed}

\section{DHKE}
The goal of this problem is to implement the Diffie Hellman Key Exchange. Below is my implementation DHKE.py:

\begin{framed}
\begin{verbatim}
import random
from square_multiply import squareMultiply
import hashlib

def isPrime(p):
    if(p==2): return True
    if(not(p&1)): return False
    return squareMultiply(2,p-1,p)==1

def rand_prime():
    while True:
        p = random.getrandbits(1023)
        q = int(p)
        if isPrime(q) == True:
            return q

def gen_Key(k_A):
    hashkey = hashlib.sha224(hex(k_A)[2:-1].decode('hex'))
    return hashkey

def checkEqual(a,b):
    if a == b:
        print "k_A = k_B"
    else:
        print "Keys not equal"

random.seed(12345)

p = rand_prime()
p_safe = int("""
136493091133649836164289363338682002944029379014077033130604363922256595
037775025761962540974136000872788770954019530710825866837156972143526820\
463536654299014569407235702375016873961943932744861043443226534544334946\
757053788912168896006231429590913701778250440452439749688803485407941150\
333770430825062090319""".replace("\n",""))

a = random.randrange(1,p_safe-2)
b = random.randrange(1,p_safe-2)
g = random.randrange(1,p_safe-1)
A = squareMultiply(g,a,p_safe)
B = squareMultiply(g,b,p_safe)

print "a= ", a,"\n\nb= ",b,"\n\np= ",p, "\n\np_safe= ", p_safe, "\n\ng= ",g, "\n"

k_A = squareMultiply(A,b,p_safe)
k_B = squareMultiply(B,a,p_safe)
checkEqual(k_A,k_B)

print "Hashed key: ",gen_Key(k_A).hexdigest()

\end{verbatim}
\end{framed}

Below is the output of DHKE.py:

\begin{framed}
\begin{lstlisting}
a=  
113932264828811031499998907582354125839977544700394975992106306
541368094048343738673665232070237154511881603757538021018841028
403063666653255093534336696687753713332940126767877373833615248
392916395921339141647849520042035488629508783383430347580636842
546861310780051238545125136953574482770599097074594064508 

b=  
126626831213495643304735514354614556800261562053466595780231899
358989161286924851285810952783577819170756527592438994158731276
746424153476456379226153790428920685854369667993183146166866881
223915207991912774025452899327038993796224713499507935329742695
616152820481902000507295224703996190730617499207101085820 

p=  
523633092592178195893526328469239769606612725211844383462007177
951575938155186456020928588747527192409767947469066041434417536
080791321590420184634380944668733233648080355859275748383165406
209569453289981241748804766181443112929482799639828141249950440
36773521761120990641451147318815085231394110302542716097 

p_safe=  
136493091133649836164289363338682002944029379014077033130604363
922256595037775025761962540974136000872788770954019530710825866
837156972143526820463536654299014569407235702375016873961943932
744861043443226534544334946757053788912168896006231429590913701
778250440452439749688803485407941150333770430825062090319 

g=  
218664716382907224032769338234721593000728013493733282508968368
268144566917238746876102897991182842205365544444737537716345776
976788062148658164102024339885546656326412043391576494706734717
366992088052216152656376977617225261050893449328066614544330277
75549180450675016451114187167720254168121847725280134920 

k_A = k_B
Hashed key:  80b3d1eed89742514603fb4998f41fd5c18a7b91869e6d040974c088
\end{lstlisting}
\end{framed}

\section{ElGamal Encryption}
The ElGamal encryption scheme is non-deterministic: A given message m has many valid encryptions.

(a) Why is the ElGamal encryption scheme non-deterministic?\\

The encryption function for the ElGamal encryption is $c^\prime = m\times h^a$. $m$ could always be the same, but the shared key $h^a = g^{ab} \bmod p$ will change as values for a and b change.\\

(b) How many valid ciphertexts exist for each message m (general expression)? How many are there for the system in problem 8.13 from the book (numerical answer)?\\

From the above equation for ciphertext encryption, we see there are $h^a = g^{ab} \bmod p$ possible ciphertexts. From 8.13:\\
$k_M = \beta^i \bmod p \textrm{ where } \beta = \alpha^d \bmod p \textrm{, }\alpha = 2 \textrm{, } p = 467$\\
1. $k_{pr} = d = 105, i = 213, x = 33$\\
2. $k_{pr} = d = 105, i = 123, x = 33$\\
3. $k_{pr} = d = 300, i = 45, x = 248$\\
4. $k_{pr} = d = 300, i = 47, x = 248$\\
We may use the following code to find the number of possible ciphertexts i.e. $k_M$:
\begin{framed}
\begin{verbatim}
from square_multiply import squareMultiply

def getKey_M(a, d, i, p):
    beta = squareMultiply(a, d, p)
    k_m = squareMultiply(beta, i, p)

    return k_m

a = 2
p = 467

arr_d = [105, 105, 300, 300]
arr_i = [213, 123, 45, 47]

for i in range(4):
    temp = getKey_M(a, arr_d[i], arr_i[i], p)
    print "Number of ciphertexts for ", i + 1, ":", temp
\end{verbatim}
\end{framed}
From this code we get the following output:
\begin{framed}
\begin{verbatim}
Number of ciphertexts for  1 : 292
Number of ciphertexts for  2 : 278
Number of ciphertexts for  3 : 12
Number of ciphertexts for  4 : 74
\end{verbatim}
\end{framed}

(c) Consider the case that for two messages $m_1 \neq m_2$ the same session parameter $y_1 = y_2$ has been chosen for the ElGamal encryption. This kind of behavior occurs if no or a bad cryptographic PRNG is being used. Show how the message $m_2$ can be recovered from a known message ciphertext pair $\langle m_1, c_1\rangle$ if the same y is used.\\

Knowing pair $\langle m_1, c_1\rangle$, we can do a search for the integer $y$ in $c_1 \equiv m_1 \times y \bmod p$. We can take the result of the search $y$ and decrypt for the pair $\langle m_2, c_2\rangle$, assuming we know a ciphertext $c_2$, using the decryption function $m_2 = c_2 \times y^{-1} \bmod p$.
\end{document}